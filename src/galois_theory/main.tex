\documentclass{jsarticle}

\title{ガロア理論12講 ノート}
\author{NS}
\begin{document}

\maketitle

\section*{序章:ガロア理論とは何か}

具体的な1-4次の解の公式が記述される。5次は見つからず、ラグランジュ、ルフィニ、アーベルが頑張った話歴史的な経緯が書かれている。

お気持ちが書かれていることを期待したが、ガチ初心者にとっては意味不明な記述しかなく、あまり役に立たない印象。

注目したい記述はあるが、
\begin{quotation}
    ガロアはラグランジュによって見出された「根の置換」による方程式論のアイデア、すなわち方程式に隠された「対称性」から、解法の可能性を解析するという新しい、そして極めて現代的な視点を打ち出した。つまり、「何かについての対称性」という見方から脱却して、対称性そのものの構造の中に重要な鍵が隠されていることを見出したのである。これが現代数学における群という考え方に繋がっている。
\end{quotation}

あとの知識から振り返ると、「根の置換」だけに注目するのではなく、\textbf{有理数並びにそれを拡大した体上での自己同型(自分自身への同型写像)全体}を考えようというところが慧眼だ、という話な気がする。 

\section{複素数と方程式}

\subsection{概要}
代数の入門(体、複素数、多項式の定義および性質など)から始めている。すでに知っている人は飛ばし読みできるが、ちょこちょこハッとする記述はある。
ハイライトは拡大と最小多項式の定義・性質の理解。振り返ってみると、以降の章で使い倒している。

\subsection{注目ポイント}

\paragraph*{冪根の添加による拡大が体になることの証明} 高校で習った「分母の有理化」の真髄がここにある。つまり、添加した元の逆元は体の元でかけるよという話。なぜ有理化すべきなのか、高校の先生に聞いた記憶があるが、明確な答えはなかった。まあ、有理数の拡大が体となることが重要! だなんて説明すべくもないとは思うが。











\end{document}