% This document is based on http://math.shinshu-u.ac.jp/~hanaki/beamer/beamer.html
\documentclass[dvipdfmx,cjk]{beamer}
%\documentclass[dvipdfm,cjk]{beamer} % オプションは環境や利用するプログラムによって変える
%\documentclass[dvips,cjk]{beamer}

\usepackage{amssymb}
% しおり(PDF にしたときの目次)の文字化け防止
\AtBeginDvi{\special{pdf:tounicode 90ms-RKSJ-UCS2}}
%\AtBeginDvi{\special{pdf:tounicode EUC-UCS2}}

% 右下のアイコンを消す
\setbeamertemplate{navigation symbols}{}

% テーマ
\usetheme{CambridgeUS}
%\usetheme{Boadilla}           %% Beamer のディレクトリの中の
%\usetheme{Madrid}             %% beamerthemeCambridgeUS.sty を指定
%\usetheme{Antibes}            %% 色々と試してみるといいだろう
%\usetheme{Montpellier}        %% サンプルが beamer\doc に色々とある。
%\usetheme{Berkeley}
%\usetheme{Goettingen}
%\usetheme{Singapore}
%\usetheme{Szeged}

%\usecolortheme{rose}          %% colortheme を選ぶと色使いが変わる
%\usecolortheme{albatross}

%\useoutertheme{shadow}                 %% 箱に影をつける
%\usefonttheme{professionalfonts}       %% 数式の文字を通常の LaTeX と同じにする

%\setbeamercovered{transparent}         %% 消えている文字をうっすらと表示する
\setbeamertemplate{theorems}[numbered]  %% 定理に番号をつける
\newtheorem{thm}{Theorem}[section]
\newtheorem{proposition}[thm]{Proposition}
\theoremstyle{example}
\newtheorem{exam}[thm]{Example}
\newtheorem{remark}[thm]{Remark}
\newtheorem{question}[thm]{Question}
\newtheorem{prob}[thm]{Problem}

\newcommand{\tr}{\text{tr}}
\newcommand{\transpose}[1]{{#1}^T}
\newcommand{\inverse}[1]{{#1}^{-1}}
\newcommand{\norm}[1]{||{#1}||}
\newcommand{\dual}[1]{{#1}^{*}}
\newcommand{\reals}{\mathbb{R}}
% メタ情報
\begin{document}
\title[Sample of Beamer]{Mini notes}
\author[nukopy]{nukopy}
\institute[nukopy org.]{nukopy の所属}
\date{February 6, 2007}

% タイトルスライド
\begin{frame}
\titlepage
\end{frame}

% 目次(\section 名が自動で挿入される)
\begin{frame}
\tableofcontents
\end{frame}

% セクション名(サンプルでは左上のスペースに表示される)
\section{Matrices}
\begin{frame}
\frametitle{Norm on matrices} % スライドのタイトル

The standard norm of vector is the sum of quadratics of componets.

What would be the matrix version? Since a matrix is also a sequence of numbers,
 the first guess is the same: the sum of quadratics of elements:

\begin{equation}
    \norm{A}^2 = \sum_{i,j} (A_{ij})^2.
\end{equation}

We can write this formula using trace:
\begin{align}
    \sum_{i,j} (A_{ij})^2 &= \sum_{i,j} \transpose{A}_{ji} A_{ij} \\
    &= \tr \transpose{A} A.
\end{align}
\end{frame}

\begin{frame}
    \frametitle{}
    It is easier to see the norm is invariant under orthogonal transformation:

    \begin{align}
        \tr \transpose{A} A &\to \tr \transpose{U A \inverse{U}} {U A \inverse{U}} \\ 
        &= \tr \transpose{\inverse{U}} \transpose{A} \transpose{U} U A \inverse{U} \\ 
        &= \tr \transpose{A} A.
    \end{align}
\end{frame}

\begin{frame}
    \frametitle{Inner product}

    By analogy, we can define inner product via vector norm.

    \begin{equation}
        (A, B) := \frac{1}{4}\left(\norm{A + B}^2 - \norm{A - B}^2 \right)
    \end{equation}

    Explicitely,
    \begin{align}
        \norm{A + B} &= \tr \transpose{\left( A+B\right)} \left( A+B\right) \\
        &= \tr \left(\transpose{A} A + \transpose{A} B + \transpose{B} A + \transpose{B} B \right) \\
        &= \norm{A} + \norm{B} + 2 \tr \transpose{A} B,
    \end{align}
    where we used the symetry property of trace:
    \begin{equation}
        \tr \transpose{A} B = \tr \transpose{B} A.
    \end{equation}

    Thus the inner product of matrices becomes
    \begin{equation}
        \left(A, B\right) = \tr \transpose{A} B.
    \end{equation}

    
\end{frame}

\begin{frame}
    \frametitle{Dual matrix}

    Inner product $(\cdot, \cdot)$ on vector space defines linear functional $\tilde{A} = (\cdot, A)$
    \begin{equation}
        (A, B) = \dual{A} B, \dual{A} = \tr \circ \transpose{A}.
    \end{equation}

\end{frame}

\begin{frame}
    \frametitle{Derivative with respect to matrix}

    We have seen that matrices can be treated as vector, and they are naturally endowed with inner product.
    By utilizing this structure, we can consider derivative of matrix-arg function quite naturally.

    Let $f: \text{Mat} \to \reals$ and $\Delta$ be an infinitesimal matrix. Then the derivative of $f$ is a matrix $L$ that satisfies the following:
    \begin{equation}
        f(A + \Delta) = f(A)  + (\Delta, L) + O(\Delta^2).
    \end{equation}

\end{frame}

% 定理
\section{定理型環境}
\begin{frame} % \newtheorem で新しい環境も作れる
\begin{thm}
定理型環境が使える。
使い方は普通の \LaTeX と同じ
\end{thm}
\pause

% 証明
\begin{proof}
証明も書ける。
\end{proof}
\pause

% 例
\begin{exam}
example
\end{exam}
\end{frame}

% 文字色を変える
\section{文字の色を変える}             %% 文字の色を変える
\begin{frame}
\frametitle{文字の色を変えてみよう!}
{\color{red}赤}\pause
{\color{blue}青}\pause
{\color{green}緑}
\end{frame}

\end{document}
