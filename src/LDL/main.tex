\documentclass[twoside, uplatex]{article}

\usepackage{aistats2023}

% \documentclass[uplatex]{article}

\usepackage{float}
\usepackage[dvipdfmx]{hyperref, graphicx}

\newcommand{\transpose}{{}^T}
\newcommand{\inverse}{{}^{-1}}
\newcommand{\half}{\frac{1}{2}}
\newcommand{\given}[2]{ {{1}|{2}}}

\title{Gaussian conditioning and LDL decomposition}
\begin{document}

\maketitle

\section{Gaussian decomposition}

\section{Gaussian elimination}

\section{LDL decomposition}

Suppose one of the diagonal block matrix is invertible. Here we assume the lower part of the block is diagonable.
Then we can eliminate upper block by applying upper triangular matrix.

\section{Conditioning Gaussian}

In general, the joint distribution is splitted into two probabilities.  One is conditional probability and the other is marginal distribution.

\begin{equation}
    \label{eq:product-rule}
    p(x_1, x_2) = p(x_1 | x_2) p(x_2).
\end{equation}

The joint distribution is written by Gaussian. We know the conditional and marginals are Gaussian as well.
Then our goal is to find their means and covariances. Let those means and covariances $mu_{\given{1}{2}}, \Sigma_{\given{1}{2}}, mu_2, \Sigma_2$.
The equation \ref{eq:product-rule} reads
\begin{equation}
    \exp \left [ -\half x\transpose \Sigma\inverse x \right ] = \exp \left [- \half x_\given{1}{2}\transpose \Sigma_\given{1}{2}\inverse x_\given{1}{2} + \half x_2 \transpose \Sigma_2 \inverse x_2 \right ]
\end{equation}
where we applied the exponential rules:  $(\exp a)(\exp b) = \exp(a+b).$


\begin{equation}
- \half  x\transpose \Sigma \inverse x = -\half \begin{bmatrix} x_1 \\ x_2 \end{bmatrix} \transpose \begin{bmatrix}\Sigma_{11} & \Sigma_{12} \\ \Sigma_{21} & \Sigma_{22} \end{bmatrix} \inverse \begin{bmatrix} x_1 \\ x_2 \end{bmatrix}.
\end{equation}

The goal is to decompose in a way that one term is pure in $x_2$, and the other is the linear mixture of $x_1$ and $x_2$.  

\begin{equation}
    \half x_\star\transpose \Sigma_\star\inverse x_\star + \half x_2 \transpose \Sigma \inverse x_2
\end{equation}

Once this decomposition is achieved we have decomposition in probability:


\end{document}