\documentclass{beamer}

\usepackage{amsmath}
\usepackage{amssymb}
\usepackage{bm}

\newcommand{\half}{\frac{1}{2}}
\newcommand{\tr}{\text{tr}}
\newcommand{\transpose}[1]{{#1}^T}
\newcommand{\inverse}[1]{{#1}^{-1}}
\newcommand{\norm}[1]{||{#1}||}
\newcommand{\dual}[1]{{#1}^{*}}
\newcommand{\reals}{\mathbb{R}}
\newcommand{\given}[2]{ {{1}|{2}}}
\newcommand{\identity}[1]{\bm{1}_n}


\begin{document}
    \begin{frame}
        \frametitle{Transaction}
    
        Transaction is charaterized by who $i$ sends which amount $v_{ij}$ to who $j$. We can encode this information in matrix:

        \begin{equation}
            V := \begin{bmatrix}
                v_{11} & \dots & v_{1n} \\
                \vdots & \dots & \vdots \\
                v_{n1} & \dots & v_{nn}
            \end{bmatrix}                
        \end{equation}

        But this representation is redundant in a sense that
        \begin{enumerate}
            \item Sending to itself $i = j$ does not make full sense. Or, the amount sent is always 0.
            \item Sending an amount from $i$ to $j$ is equivalent to sending the \emph{negative amount} from $j$ to $i$.
        \end{enumerate}
        In matrix notation, those redundancy are charaterized by the antisymmetry of $V$: $\transpose{V} = - V.$

    \end{frame}

    \begin{frame}
        \frametitle{Toward the Law of Transaction}
    
        The economy is driven by a series of transaction.
        It makes sense that we consider the trajectory of transaction along time $t$. 
        We assume transaction does not take place randomly, as suggested by economics, game theory , etc.
        Here we focus on deterministic behaviour. In this case our system should be endowed with some ordinary differential equation (ODE):
        \begin{equation}
            \frac{d V}{d t} = K(t, V(t), \dots),
        \end{equation}
        where $K$ is an unknown function of time $t$, $V$, and possibly some other variables.

    \end{frame}
\end{document}